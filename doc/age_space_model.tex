% LyX 2.2.4 created this file.  For more info, see http://www.lyx.org/.
%% Do not edit unless you really know what you are doing.
\documentclass[english]{article}
\usepackage[T1]{fontenc}
\usepackage[latin9]{inputenc}
\usepackage{geometry}
\geometry{verbose,tmargin=2.5cm,bmargin=2.5cm,lmargin=2.5cm,rmargin=2.5cm}
\setlength{\parskip}{\medskipamount}
\setlength{\parindent}{0pt}
\usepackage{url}
\usepackage{bm}
\usepackage{amsmath}
\usepackage{babel}
\begin{document}

\title{UK Age and Space structured Covid-19 model}

\author{Chris Jewell, Barry Rowlingson, Jon Read}
\maketitle

\section{Concept}

We wish to develop a model that will enable us to assess spatial spread
of Covid-19 across the UK, respecting the availability of human mobility
data as well as known contact behaviour between individuals of different
ages.

A deterministic SEIR state transition model is posited in which individuals
transition from Susceptible to Exposed (i.e. infected but not yet
infectious) to Infectious to Removed (i.e. quarantined, got better,
or died). 

We model the infection rate (rate of S$\rightarrow$E transition)
as a function of known age-structured contact from Polymod, known
human mobility between MSOAs (Middle Super Output Area), and Census-derived
age structured population density in regions across the UK.

Currently, this model is populated with data for England only, though
we are in the process of extending this to Scotland, Wales, and Northern
Ireland.

\section{Data}

\subsection{Age-mixing}

Standard Polymod social mixing data for the UK are used, with 17 5-year
age groups $[0-5),[5-10),\dots,[75-80),[80-\infty)$. Estimated contact
matrices for term-time $M_{tt}$ and school-holidays $M_{hh}$ were
extracted of dimension $n_{m}\times n_{m}$ where $n_{m}=17$.

\subsection{Human mobility}

2011 Census data from ONS on daily mean numbers of commuters moving
from each Residential MSOA to Workplace MSOA. MSOAs are aggregated
to Local Authority Districts (LADs) for which we have age-structured
population density. The resulting matrix $C$ is of dimension $n_{c}\times n_{c}$
where $n_{c}=152$. Since this matrix is for Residence to Workplace
movement only, we assume that the mean number of journeys between
each LAD is given by
\[
T=C+C^{T}
\]
with 0 diagonal.

\subsection{Population size}

Age-structured population size within each LAD is taken from publicly
available 2019 Local Authority data giving a vector $N$ of length
$n_{m}n_{c}=2584$, i.e. population for each of $n_{m}$ age groups
and $n_{c}$ LADs.

\section{Model}

\subsection{Connectivity matrix}

We assemble a country-wide connectivity matrices as Kronecker products,
such that
\[
M^{\star}=I_{n_{c}}\bigotimes M
\]
 and
\[
C^{\star}=C\bigotimes\bm{1}_{n_{m}\times n_{c}}
\]
giving two matrices of dimension $n_{m}n_{c}\times n_{m}n_{c}$. $M^{\star}$
is block diagonal with Polymod mixing matrices. $C^{\star}$ expands
the mobility matrix $C$ such that a block structure of connectivity
between LADs results.

\subsection{Disease progression model}

We assume an SEIR model described as a system of ODEs. We denote the
number of individual in each age-group-LAD combination at time $t$
by the vectors $\vec{S}(t),\vec{E}(t),\vec{I}(t),\vec{R}(t)$. We
therefore have
\begin{align*}
\frac{\mathrm{d\vec{S}(t)}}{dt} & =-\beta_{t}\left[M^{\star}\vec{I}(t)+\beta_{2}w_{t}\bar{M}C^{\star}\frac{{\vec{I}(t)}}{N}\right]\frac{\vec{S}(t)}{N}\\
\frac{\mathrm{d}\vec{E}(t)}{dt} & =\beta_{t}\left[M^{\star}\vec{I}(t)+\beta_{2}w_{t}\bar{M}C^{\star}\frac{{\vec{I}(t)}}{N}\right]\frac{\vec{S}(t)}{N}-\nu\vec{E}(t)\\
\frac{\mathrm{d}\vec{I}(t)}{dt} & =\nu\vec{E}(t)-\gamma\vec{I}(t)\\
\frac{\mathrm{d}\vec{R}(t)}{dt} & =\gamma\vec{I}(t)
\end{align*}
where $\bar{M}$ is the global mean person-person contact rate, and
$w_{t}$ is the total rail ticket sales in the UK expressed as a fraction
of the 2019 mean (a proxy for reduction in travel). Parameters are: 

\begin{align*}
\beta_{t} & =\begin{cases}
\beta_{1} & \mbox{if }t<T_{L}\\
\beta_{1}\beta_{3} & \mbox{otherwise }
\end{cases}\\
\end{align*}
with $T_{L}$ the date of lock-down restrictions, $\beta_{1}$ a baseline
transmisison rate, and $\beta_{3}$ giving the ratio of post-lockdown
to pre-lockdown transmission; commuting infection ratio $\beta_{2}$;
latent period $\frac{1}{\nu}=4$days; and infectious period $\frac{1}{\gamma}$.
Typically, we assume that contact with commuters is $\beta_{2}=\frac{1}{3}$
of that between members of the same age-LAD combination assuming an
8 hour working day.

\subsection{Noise model}

Currently, and subject to discussion, we assume that all detected
cases are synonymous with individuals transitioning $I\rightarrow R$.
We assume the number of new cases in each age-LAD combination are
given by
\[
y_{ik}(t)\sim\mbox{Negative Binomial}\left(r,\phi(R_{ik}(t)-R_{ik}(t-1))\right)
\]

where $\phi$ is the case reporting fraction (i.e. proportion of infections
that are eventually detected) and $r$ is an overdispersion parameter.

\subsection{Inference}

We are interested in making inference on $\epsilon_{0},\beta_{1}$,
and $\gamma$. Priors are currently:
\begin{align*}
\epsilon_{0} & \sim\mbox{Gamma}(1,1)\\
\beta_{1} & \sim\mbox{Gamma}(1,1)\\
\beta_{3} & \sim\mbox{Gamma}(20,20)\\
\gamma & \sim\mbox{Gamma}(10,40)\\
r & \sim\mbox{\mbox{Gamma}(}0.1,0.1)
\end{align*}
specified to express relative ignorance about $\epsilon_{0}$ , $\beta_{1}$,
and $r$, strong information on $\beta_{3}$ and $r$, and a belief
that the infectious period should be approximately 4 days.

\subsection{Implementation}

The model is currently implemented in Python3, using Tensorflow 2.1
with the RK5 differential equation solver implemented in the \texttt{DormandPrince}
class provided by Tensorflow Probability 0.9. Code may be found at
\url{http://fhm-chicas-code.lancs.ac.uk/jewell/covid19uk}.

The code is a work in progress \textendash{} see README.

\section{Results}
\end{document}
